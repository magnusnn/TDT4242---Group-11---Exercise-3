\section{Tests and Testing Result}

This section will presents the test cases and their results. 

\subsection{Manufacturer system}

This section will present the requirement tests of the manufacturer system.
Note, all test plans requires that the user is logged into the factory system.

\begin{table}[H]
\centering
\caption{Manufacturer system test}
\begin{tabularx}{1.0\textwidth}{
    |p{\dimexpr.25\linewidth-2\tabcolsep-1.3333\arrayrulewidth}     % column 1
    |p{\dimexpr.75\linewidth-2\tabcolsep-1.3333\arrayrulewidth}|    % column 2
}
\hline

Test ID
& 1.1
\\
\hline

Test name
& Add New Software Version
\\
\hline

Main goal
& Insert new software and new versions of existing software into a software database
\\
\hline

Requirement(s)
& The new software and new versions of software the CM enters shall be stored in a software archive
\\
\hline

Description
& All new software versions that are created by the CM should be stored in the software archive
\\
\hline

Stakeholders
& End-users, developers, customer, testers
\\
\hline

Risk
& The  software  entered  by  the  CM  into  the  software  archive can be erroneous.
\\
\hline

Test type
& System
\\
\hline

Preconditions
& -
\\
\hline

Input
& \begin{enumerate}
    \item Major version: 6
    \item Fetch URL: sw6.0.net
    \item Press 'Save' button
\end{enumerate}
\\
\hline

Output
& \begin{enumerate}
    \item Minor version: 0
    \item Status: Software added
    \item 6(0) in Software Archive dropdown list
\end{enumerate}
\\
\hline

Expected result
& Status: Software added
\\
\hline

Result
& Status: Software added
\\
\hline

Comments
& Adding a new software component will add it into the archive with the given version number successfully. Though, adding an existing software component will yield a status of it already existing in the archive, and instead add a new sub id to that version.
\\
\hline

Passed
& Yes
\\
\hline

\end{tabularx}
\end{table}

% ---------------- %
\begin{table}[H]
\centering
\caption{Manufacturer system test}
\begin{tabularx}{1.0\textwidth}{
    |p{\dimexpr.25\linewidth-2\tabcolsep-1.3333\arrayrulewidth}     % column 1
    |p{\dimexpr.75\linewidth-2\tabcolsep-1.3333\arrayrulewidth}|    % column 2
}
\hline

Test ID
& 1.2
\\
\hline

Test name
& Unique Part Number
\\
\hline

Main goal
& Insert new software and new versions of existing software into a software database
\\
\hline

Requirement(s)
& The software shall version the software components by subnumbering the unique part number
\\
\hline

Description
& It should be possible to identify every software component in the archive by their unique part number. The system shall version the software components by the use of subnumbering. 
\\
\hline

Stakeholders
& End-users, developers, customer, testers
\\
\hline

Risk
& The software entered into the software archive does not receive the correct subnumber for versioning
\\
\hline

Test type
& System
\\
\hline

Preconditions
& An earlier existing software component version
\\
\hline

Input
& \begin{enumerate}
    \item Major version: 6
    \item Fetch URL: sw6.1.net
    \item Press 'Save' button
\end{enumerate}
\\
\hline

Output
& \begin{enumerate}
    \item Minor version: 1
    \item Status: Software added
    \item 6(1) in Software Archive dropdown list
\end{enumerate}
\\
\hline

Expected result
& Status: Software id already in db, added new sub id
\\
\hline

Result
& Status: Software id already in db, added new sub id
\\
\hline

Comments
& Adding an already existing software component will return with Status: Software id already in db, added new sub id. The software component will be added to the archive with a new minor version in parenthesis. 
\\
\hline

Passed
& Yes
\\
\hline

%note

\end{tabularx}
\end{table}

% ---------------- %
\begin{table}[H]
\centering
\caption{Manufacturer system test}
\begin{tabularx}{1.0\textwidth}{
    |p{\dimexpr.25\linewidth-2\tabcolsep-1.3333\arrayrulewidth}     % column 1
    |p{\dimexpr.75\linewidth-2\tabcolsep-1.3333\arrayrulewidth}|    % column 2
}
\hline

Test ID
& 1.3
\\
\hline

Test name
& Add Action Scripts to Database
\\
\hline

Main goal
& Maintain a database that shows which control units (ECU)that needs which software
\\
\hline

Requirement(s)
& The  action  scripts  the  CM  enters  shall  be  inserted  in  the action database
\\
\hline

Description
& It should be able to create an action script by the CM that should be saved in the action database
\\
\hline

Stakeholders
& End-users, developers, customer, testers
\\
\hline

Risk
& The action script the CM enters can be faulty
\\
\hline

Test type
& System
\\
\hline

Preconditions
& Software component in archive
\\
\hline

Input
& \begin{enumerate}
    \item Press New button
    \item ECU ID: 4
    \item Software ID: 6
    \item Press Save button
\end{enumerate}
\\
\hline

Output
& \begin{enumerate}
    \item Status: ECU added to database
\end{enumerate}
\\
\hline

Expected result
& Status: Ecu added to database
\\
\hline

Result
& Status: Ecu added to database
\\
\hline

Comments
& Adding a new ECU is done by adding a new ECU ID, if it is already in the database the output will say so. Then, adding a software ID existing from the archive, if that already is connecting to another ECU the output will say so. If adding a new ECU ID with a free software component, they will be binded and added to the database.
\\
\hline

Passed
& Yes
\\
\hline

\end{tabularx}
\end{table}

% ---------------- %
\begin{table}[H]
\centering
\caption{Manufacturer system test}
\begin{tabularx}{1.0\textwidth}{
    |p{\dimexpr.25\linewidth-2\tabcolsep-1.3333\arrayrulewidth}     % column 1
    |p{\dimexpr.75\linewidth-2\tabcolsep-1.3333\arrayrulewidth}|    % column 2
}
\hline

Test ID
& 1.4
\\
\hline

Test name
& Update Action Script
\\
\hline

Main goal
& Maintain a database that shows which control units (ECU)that needs which software
\\
\hline

Requirement(s)
& The action scripts shall be updatable by the CM
\\
\hline

Description
& The CM should be able to update every action script in the database
\\
\hline

Stakeholders
& End-users, developers, customer, testers
\\
\hline

Risk
& The updated action script can be faulty
\\
\hline

Test type
& System
\\
\hline

Preconditions
& An existing action script
\\
\hline

Input
& -
\\
\hline

Output
& -
\\
\hline

Expected result
& Status: Action script updated
\\
\hline

Result
& -
\\
\hline

Comments
& It doesn't seem like it is possible to update an action script within the system. Neither is it possible to delete created action script, and remake it.
\\
\hline

Passed
& No
\\
\hline

\end{tabularx}
\end{table}

% ---------------- %
\begin{table}[H]
\centering
\caption{Manufacturer system test}
\begin{tabularx}{1.0\textwidth}{
    |p{\dimexpr.25\linewidth-2\tabcolsep-1.3333\arrayrulewidth}     % column 1
    |p{\dimexpr.75\linewidth-2\tabcolsep-1.3333\arrayrulewidth}|    % column 2
}
\hline

Test ID
& 1.5
\\
\hline

Test name
& Software Component Belonging to ECU
\\
\hline

Main goal
& Maintain a database that shows which control units (ECU)that needs which software
\\
\hline

Requirement(s)
& The action script shall give information of which ECU needs which software component
\\
\hline

Description
& It should be able to retrieve correct information of which ECU that need which software component from the action script
\\
\hline

Stakeholders
& End-users, developers, customer, testers
\\
\hline

Risk
& The action script gives the wrong information on which ECU needs which software component
\\
\hline

Test type
& System
\\
\hline

Preconditions
& Active action script
\\
\hline

Input
& \begin{enumerate}
    \item Choosing ECU from the dropdown list
    \item ECU: 2
\end{enumerate}
\\
\hline

Output
& \begin{enumerate}
    \item The rows underneath the dropdown list shows which ECU is connected to the software component
    \item ECU ID: 2, Software ID: 2
\end{enumerate}
\\
\hline

Expected result
& Finding which ECU is connected to which software
\\
\hline

Result
& Found out which ECU is connected to which software
\\
\hline

Comments
& It is possible to see which ECU is connected to which software by choosing any ECU from the dropdown list. The chosen ECU will show in the row underneath the dropdown list, and underneath there again the software connected to that ECU is displayed. 
\\
\hline

Passed
& Yes
\\
\hline

\end{tabularx}
\end{table}

% ---------------- %
\begin{table}[H]
\centering
\caption{Manufacturer system test}
\begin{tabularx}{1.0\textwidth}{
    |p{\dimexpr.25\linewidth-2\tabcolsep-1.3333\arrayrulewidth}     % column 1
    |p{\dimexpr.75\linewidth-2\tabcolsep-1.3333\arrayrulewidth}|    % column 2
}
\hline

Test ID
& 1.6
\\
\hline

Test name
& Connection Between ECU and Software
\\
\hline

Main goal
& Maintain a database that shows which control units (ECU)that needs which software
\\
\hline

Requirement(s)
& The action script shall connect the ECU to the software component that is used to control it
\\
\hline

Description
& Created action scripts should connect ECU to software component that is used to control it.
\\
\hline

Stakeholders
& End-users, developers, customer, testers
\\
\hline

Risk
& The action script connects the ECU to the wrong software component
\\
\hline

Test type
& System
\\
\hline

Preconditions
& There must be a software version in the software archive
\\
\hline

Input
& \begin{enumerate}
    \item Press New button
    \item ECU ID: 4
    \item Software ID: 6
    \item Press Save button
\end{enumerate}
\\
\hline

Output
& \begin{enumerate}
    \item Status: ECU added to the database
    \item ECU is now connected to the software
\end{enumerate}
\\
\hline

Expected result
& ECU connecting to software
\\
\hline

Result
& ECU connected to software
\\
\hline

Comments
& When creating a new action script with a new ECU and a software component that is not connected to another ECU, the new ECU will automatically connect to that software component. If that software component is already connected to another ECU, the output will yield an error message, 'Status: The software is controlling another ecu, please enter a new one'.
\\
\hline

Passed
& Yes
\\
\hline

\end{tabularx}
\end{table}

% ---------------- %
\begin{table}[H]
\centering
\caption{Manufacturer system test}
\begin{tabularx}{1.0\textwidth}{
    |p{\dimexpr.25\linewidth-2\tabcolsep-1.3333\arrayrulewidth}     % column 1
    |p{\dimexpr.75\linewidth-2\tabcolsep-1.3333\arrayrulewidth}|    % column 2
}
\hline

Test ID
& 1.7
\\
\hline

Test name
& Add New Series to Vehicle Database
\\
\hline

Main goal
& Assist the garage in updating car software
\\
\hline

Requirement(s)
& The  entry  for  a  new  production  series  defined  by  the  CM shall be inserted into the vehicle database
\\
\hline

Description
& All new production series that are added by the CM should be added into the vehicle database
\\
\hline

Stakeholders
& End-users, developers, customer, testers
\\
\hline

Risk
& The vehicle database entry may contain wrong information
\\
\hline

Test type
& System
\\
\hline

Preconditions
& 
\\
\hline

Input
& \begin{enumerate}
    \item Press New button
    \item Vehicle ID: 6
    \item Series: 200
    \item ECU ID: 2, Major SW version: 2, Minor SW version: 1
    \item Press Save button
\end{enumerate}
\\
\hline

Output
& \begin{enumerate}
    \item Status: Vehicle added to database
\end{enumerate}
\\
\hline

Expected result
& Status: Vehicle added to database
\\
\hline

Result
& Status: Vehicle added to database
\\
\hline

Comments
& Adding a new series to the vehicle database is successful when adding a new unique vehicle ID with the new vehicle series. If the vehicle ID already exists, the output will say so, it will still add it to the database and increment the vehicle ID by one.
\\
\hline

Passed
& Yes
\\
\hline

\end{tabularx}
\end{table}

% ---------------- %
\begin{table}[H]
\centering
\caption{Manufacturer system test}
\begin{tabularx}{1.0\textwidth}{
    |p{\dimexpr.25\linewidth-2\tabcolsep-1.3333\arrayrulewidth}     % column 1
    |p{\dimexpr.75\linewidth-2\tabcolsep-1.3333\arrayrulewidth}|    % column 2
}
\hline

Test ID
& 1.8
\\
\hline

Test name
& Identify Faulty Vehicles
\\
\hline

Main goal
& Send an alarm (”recall”) if we discover critical defect
\\
\hline

Requirement(s)
& Through the system the CM shall be able to identify all the vehicles with faulty components
\\
\hline

Description
& It should be possible for the CM through the system to identify every vehicle that have one or more faulty components
\\
\hline

Stakeholders
& End-users, developers, customer, testers
\\
\hline

Risk
& The faulty component of one or more vehicles is not identified
\\
\hline

Test type
& System
\\
\hline

Preconditions
& Faulty vehicle
\\
\hline

Input
& \begin{enumerate}
    \item Choose vehicle from dropdown list in vehicle database
    \item See history log for vehicle fault
    \item Choose vehicle 4
\end{enumerate}
\\
\hline

Output
& \begin{enumerate}
    \item Vehicle information displayed
    \item Type of fault
    \item Vehicle fault, wrecked
\end{enumerate}
\\
\hline

Expected result
& See all vehicles with software faults
\\
\hline

Result
& Not able to see vehicles with only software faults
\\
\hline

Comments
& Displaying vehicle faults is done by checking the vehicle database, it will list the fault in history log. It is not possible to get a list of software faults of each vehicle.
\\
\hline

Passed
& Yes
\\
\hline

\end{tabularx}
\end{table}

% ---------------- %
\begin{table}[H]
\centering
\caption{Manufacturer system test}
\begin{tabularx}{1.0\textwidth}{
    |p{\dimexpr.25\linewidth-2\tabcolsep-1.3333\arrayrulewidth}     % column 1
    |p{\dimexpr.75\linewidth-2\tabcolsep-1.3333\arrayrulewidth}|    % column 2
}
\hline

Test ID
& 1.9
\\
\hline

Test name
& Send Information to Garages
\\
\hline

Main goal
& Send an alarm (”recall”) if we discover critical defect
\\
\hline

Requirement(s)
& The CM shall be able to send information plus the new version of all components to all registered garages
\\
\hline

Description
& It should be possible for the CM to send information as well the new version of software components to all registered garages
\\
\hline

Stakeholders
& End-users, developers, customer, testers
\\
\hline

Risk
& Data is not sent to all registered garages
\\
\hline

Test type
& System
\\
\hline

Preconditions
& There must be a major software version in the software archive
\\
\hline

Input
& \begin{enumerate}
    \item Major software version: 6
    \item Minor software version: 1
    \item Press Get list button
    \item Press Send button
\end{enumerate}
\\
\hline

Output
& \begin{enumerate}
    \item Status: You may now push the 'Send' button in order to recall vehicles
    \item Status: Emails sent!
\end{enumerate}
\\
\hline

Expected result
& Status: Emails sent!
\\
\hline

Result
& Status: Emails sent!
\\
\hline

Comments
& Sending information to the garages with the the version of software component is done by entering the appropriate system version, then pressing get list to get the list of vehicles and send that list to garages. Though, entering a non-existing software component version and sending that to garages will yield no errors and still send the emails. It is not possible to send additional information.
\\
\hline

Passed
& Yes
\\
\hline


\end{tabularx}
\end{table}
% ---------------- %


\clearpage

\subsection{Garage system}

This section will present the requirement tests of the garage system.


\begin{table}[H]
\centering
\caption{Garage system test}
\begin{tabularx}{1.0\textwidth}{
    |p{\dimexpr.25\linewidth-2\tabcolsep-1.3333\arrayrulewidth}     % column 1
    |p{\dimexpr.75\linewidth-2\tabcolsep-1.3333\arrayrulewidth}|    % column 2
}
\hline

Test ID
& 2.1
\\
\hline

Test name
& Customer information and vehicle serial number
\\
\hline

Main goal
& Customer information and vehicle serial number should be present for each car produced
\\
\hline

Requirement(s)
& All cars produced shall have an associated serial number and customer information
\\
\hline

Description
& All cars produced shall have an associated serial number and customer information
\\
\hline

Stakeholders
& End-users, developers, customer, testers
\\
\hline

Risk
& The system has the wrong/no information present
\\
\hline

Test type
& System
\\
\hline

Preconditions
& Customer and car database must be populated with atleast one entry for both
\\
\hline

Input
& \begin{enumerate}
    \item Select an owner
\end{enumerate}
\\
\hline

Output
& \begin{enumerate}
    \item Customer and Car view populates with correct data (customer contact information and vehicle serial number)
\end{enumerate}
\\
\hline

Expected result
& Customer and vehicle data is loaded
\\
\hline

Result
& Customer and vehicle data is loaded
\\
\hline

Comments
& Everything loads as expected after connecting the garage program to the factory server, or by loading the local database.
\\
\hline

Passed
& Yes
\\
\hline

\end{tabularx}
\end{table}

% ---------------- %
\begin{table}[H]
\centering
\caption{Garage system test}
\begin{tabularx}{1.0\textwidth}{
    |p{\dimexpr.25\linewidth-2\tabcolsep-1.3333\arrayrulewidth}     % column 1
    |p{\dimexpr.75\linewidth-2\tabcolsep-1.3333\arrayrulewidth}|    % column 2
}
\hline

Test ID
& 2.2
\\
\hline

Test name
& Download vehicle information from main computer
\\
\hline

Main goal
& Information associated with a vehicle is downloaded when a vehicle serial number is provided as a key.
\\
\hline

Requirement(s)
& Data is downloaded when a valid serial number is supplied
\\
\hline

Description
& Information associated with a vehicle is downloaded when a vehicle serial number is provided as a key.
\\
\hline

Stakeholders
& End-users, developers, customer, testers
\\
\hline

Risk
& Wrong serial number is entered, returning data from another vehicle than intended.
\\
\hline

Test type
& System
\\
\hline

Preconditions
& At least one vehicle record must be present and the garage software needs to be connected to the factory central server
\\
\hline

Input
& \begin{enumerate}
    \item Select "Search VehicleID" from tools.
    \item Input vehicle ID number and press "OK".
\end{enumerate}
\\
\hline

Output
& \begin{enumerate}
    \item Box in which to enter serial number opens
    \item View updates with the vehicle's registered owner selected.
\end{enumerate}
\\
\hline

Expected result
& Customer and vehicle data is loaded when serial number is entered.
\\
\hline

Result
& Customer and vehicle data is loaded when serial number is entered.
\\
\hline

Comments
& Everything loads as expected after connecting the garage program to the factory server. Entering an erronus/wrong serial number returns the data associated with that serial number. Entering a non-existent serial number logs an exception in the serial log, and continues running without problems, no owner is selected and the views don't update.
\\
\hline

Passed
& Yes
\\
\hline

\end{tabularx}
\end{table}

% ---------------- %

\begin{table}[H]
\centering
\caption{Garage system test}
\begin{tabularx}{1.0\textwidth}{
    |p{\dimexpr.25\linewidth-2\tabcolsep-1.3333\arrayrulewidth}     % column 1
    |p{\dimexpr.75\linewidth-2\tabcolsep-1.3333\arrayrulewidth}|    % column 2
}
\hline

Test ID
& 2.3
\\
\hline

Test name
& Dual configuration information for ECU.
\\
\hline

Main goal
& The system shall use two sets of configuration information to to identify the latest version of each software component for each ECU in the vehicle
\\
\hline

Requirement(s)
& Correct dual ECU information is retrieved
\\
\hline

Description
& Each vehicle has two sets of ECU configuration information stored, both needs to be retrieved and displayed when car data is downloaded from the central server.
\\
\hline

Stakeholders
& End-users, developers, customer, testers
\\
\hline

Risk
& Software version for one or more component cannot be identified or differs from the last version.
\\
\hline

Test type
& System
\\
\hline

Preconditions
& At least one vehicle record must be present either in the local or central database
\\
\hline

Input
& \begin{enumerate}
    \item Select a vehicle and press "Get Vehicle Information"
\end{enumerate}
\\
\hline

Output
& \begin{enumerate}
    \item View updates with the vehicle's dual ECU information. Show the installed ECU software version, and whether or not it is the latest minor version.
\end{enumerate}
\\
\hline

Expected result
& Dual ECU information is downloaded and displayed correctly.
\\
\hline

Result
& Dual ECU information is downloaded and displayed correctly.
\\
\hline

Comments
& Everything loads as expected after connecting the garage program to the factory server. The information displayed is correct. Adding a new minor ECU version in the factory, and then reloading the garage view correctly indicates that a new software is available for the given car and that an update needs to be installed.
\\
\hline

Passed
& Yes
\\
\hline

\end{tabularx}
\end{table}

% ---------------- %

% ---------------- %

\begin{table}[H]
\centering
\caption{Garage system test}
\begin{tabularx}{1.0\textwidth}{
    |p{\dimexpr.25\linewidth-2\tabcolsep-1.3333\arrayrulewidth}     % column 1
    |p{\dimexpr.75\linewidth-2\tabcolsep-1.3333\arrayrulewidth}|    % column 2
}
\hline

Test ID
& 2.4
\\
\hline

Test name
& Download latest version of all non-installed components.
\\
\hline

Main goal
& The system shall download the latest version of all components that are needed for the vehicle but not yet installed.
\\
\hline

Requirement(s)
& Download the latest version of all non-installed components for a given car.
\\
\hline

Description
& When vehicle information is retrieved, download any needed outdated/non-installed components.
\\
\hline

Stakeholders
& End-users, developers, customer, testers
\\
\hline

Risk
& The system downloads and installs the wrong software version
\\
\hline

Test type
& System
\\
\hline

Preconditions
& Updated software components need to be available.
\\
\hline

Input
& \begin{enumerate}
    \item Select a vehicle with outdated components and press "Get Vehicle Information"
\end{enumerate}
\\
\hline

Output
& \begin{enumerate}
    \item View updates with the vehicle's dual ECU information. View shows that ECU component is outdated. New ECU version is downloaded but not installed
\end{enumerate}
\\
\hline

Expected result
& New components are downloaded when an outdated component is detected.
\\
\hline

Result
& New components are downloaded when an outdated component is detected.
\\
\hline

Comments
& Everything loads as expected after connecting the garage program to the factory server. View indicates that one or more ECU components are out of date, and new ones are downloaded but not installed, as expected. There's little risk of wrong information being displayed or wrong components are downloaded, as long as the correct vehicle ID has been entered. Factory side problems can lead to wrong ECU software to be installed on a non-compatible unit.
\\
\hline

Passed
& Yes
\\
\hline

\end{tabularx}
\end{table}

% ---------------- %

\begin{table}[H]
\centering
\caption{Garage system test}
\begin{tabularx}{1.0\textwidth}{
    |p{\dimexpr.25\linewidth-2\tabcolsep-1.3333\arrayrulewidth}     % column 1
    |p{\dimexpr.75\linewidth-2\tabcolsep-1.3333\arrayrulewidth}|    % column 2
}
\hline

Test ID
& 2.5
\\
\hline

Test name
& Install new software components
\\
\hline

Main goal
& The system shall install all new software components.
\\
\hline

Requirement(s)
& Install new/updated software for out of that components.
\\
\hline

Description
& When a vehicle with out of date components are selected and updated, new software is installed and the updated information is sent to the central server.
\\
\hline

Stakeholders
& End-users, developers, customer, testers
\\
\hline

Risk
& One or more software component installation fails.
\\
\hline

Test type
& System
\\
\hline

Preconditions
& New installable software component needs to be present, and a compatible vehicle.
\\
\hline

Input
& \begin{enumerate}
    \item Select a vehicle with outdated components and press "Get Vehicle Information"
    \item Press "Update car"
\end{enumerate}
\\
\hline

Output
& \begin{enumerate}
    \item View updates with the vehicle's dual ECU information. View shows that ECU component is outdated.
    \item Out of data components are updated/installed. History log is automatically amended with correct date and description of the changes. Out of date flag is removed from view. Updated information is sent to central server.
\end{enumerate}
\\
\hline

Expected result
& Out of date components are upgraded with new software.
\\
\hline

Result
& Out of date components are upgraded with new software.
\\
\hline

Comments
& Everything loads as expected after connecting the garage program to the factory server. New components are installed correctly, and the central server is notified of the changes. There's little risk of wrong information being displayed or wrong components are installed, as long as the correct vehicle ID has been entered. Factory side problems can lead to wrong ECU software to be installed on a non-compatible unit.
\\
\hline

Passed
& Yes
\\
\hline

\end{tabularx}
\end{table}

% ---------------- %

\begin{table}[H]
\centering
\caption{Garage system test}
\begin{tabularx}{1.0\textwidth}{
    |p{\dimexpr.25\linewidth-2\tabcolsep-1.3333\arrayrulewidth}     % column 1
    |p{\dimexpr.75\linewidth-2\tabcolsep-1.3333\arrayrulewidth}|    % column 2
}
\hline

Test ID
& 2.6
\\
\hline

Test name
& Update system history log
\\
\hline

Main goal
& The system shall update the central history database on changes
\\
\hline

Requirement(s)
& The system shall update the history log and send it to the action database
\\
\hline

Description
& Whenever a change is performed on a car the history log is appended, and sent to the central server.
\\
\hline

Stakeholders
& End-users, developers, customer, testers
\\
\hline

Risk
& The updated history log is erroneus
\\
\hline

Test type
& System
\\
\hline

Preconditions
& The system needs to be connected to the central database and at least one vehicle record must be present in the database.
\\
\hline

Input
& \begin{enumerate}
    \item Select a vehicle
    \item Perform a change to the vehicle details
    \item Commit the changes
\end{enumerate}
\\
\hline

Output
& \begin{enumerate}
    \item View updates with vehicle information
    \item New information is stored in the garage history log
    \item Changes are committed in full to the central database
\end{enumerate}
\\
\hline

Expected result
& History log is updated on the central server
\\
\hline

Result
& History log is updated on the central server
\\
\hline

Comments
& Everything loads as expected after connecting the garage program to the factory server. Any changes to the vehicle log in the garage view is commited to the central database
\\
\hline

Passed
& Yes
\\
\hline

\end{tabularx}
\end{table}

% ---------------- %

\begin{table}[H]
\centering
\caption{Garage system test}
\begin{tabularx}{1.0\textwidth}{
    |p{\dimexpr.25\linewidth-2\tabcolsep-1.3333\arrayrulewidth}     % column 1
    |p{\dimexpr.75\linewidth-2\tabcolsep-1.3333\arrayrulewidth}|    % column 2
}
\hline

Test ID
& 2.7
\\
\hline

Test name
& Update vehicle configuration information
\\
\hline

Main goal
& The system shall update vehicle configuration information
\\
\hline

Requirement(s)
& The system shall update vehicle configuration information
\\
\hline

Description
& The system updates the vehicle configuration information
\\
\hline

Stakeholders
& End-users, developers, customer, testers
\\
\hline

Risk
& The updated vehicle's configuration is faulty or the vehicle's configuration is not updates.
\\
\hline

Test type
& System
\\
\hline

Preconditions
& A vehicle must be in the local or central database
\\
\hline

Input
& \begin{enumerate}
    \item Select a vehicle
    \item Perform a change to the vehicle information
    \item Commit the changes
\end{enumerate}
\\
\hline

Output
& \begin{enumerate}
    \item View updates with vehicle information
    \item New information is stored
    \item Changes are committed in full to the garage database
\end{enumerate}
\\
\hline

Expected result
& Vehicle configuration information is updated
\\
\hline

Result
& Vehicle configuration information is updated
\\
\hline

Comments
& Everything loads as expected after connecting the garage program to the factory server. Any changes to the vehicle configuration in the garage view is committed to the garage database
\\
\hline

Passed
& Yes
\\
\hline

\end{tabularx}
\end{table}

% ---------------- %

\begin{table}[H]
\centering
\caption{Garage system test}
\begin{tabularx}{1.0\textwidth}{
    |p{\dimexpr.25\linewidth-2\tabcolsep-1.3333\arrayrulewidth}     % column 1
    |p{\dimexpr.75\linewidth-2\tabcolsep-1.3333\arrayrulewidth}|    % column 2
}
\hline

Test ID
& 2.8
\\
\hline

Test name
& Update vehicle configuration information in central database.
\\
\hline

Main goal
& The system shall update vehicle's configuration in the vehicle database.
\\
\hline

Requirement(s)
& The system shall update vehicle's configuration in the vehicle database.
\\
\hline

Description
& The system updates the vehicle configuration information in the central database.
\\
\hline

Stakeholders
& End-users, developers, customer, testers
\\
\hline

Risk
& The vehicle database is updated with faulty configuration or not updated with the vehicle configuration information.
\\
\hline

Test type
& System
\\
\hline

Preconditions
& The garage software needs to be connected to the central database and atleast one vehicle record must be present.
\\
\hline

Input
& \begin{enumerate}
    \item Select a vehicle
    \item Perform a change to the vehicle information
    \item Commit the changes
\end{enumerate}
\\
\hline

Output
& \begin{enumerate}
    \item View updates with vehicle information
    \item New information is stored
    \item Changes are committed in full to the central database
\end{enumerate}
\\
\hline

Expected result
& Vehicle configuration information is updated in the central database
\\
\hline

Result
& Vehicle configuration information is updated in the central database
\\
\hline

Comments
& Everything loads as expected after connecting the garage program to the factory server. Any changes to the vehicle configuration in the garage view is committed to the central database
\\
\hline

Passed
& Yes
\\
\hline

\end{tabularx}
\end{table}

% ---------------- %

\begin{table}[H]
\centering
\caption{Garage system test}
\begin{tabularx}{1.0\textwidth}{
    |p{\dimexpr.25\linewidth-2\tabcolsep-1.3333\arrayrulewidth}     % column 1
    |p{\dimexpr.75\linewidth-2\tabcolsep-1.3333\arrayrulewidth}|    % column 2
}
\hline

Test ID
& 2.9
\\
\hline

Test name
& Map Customer to vehicle and vice-versa
\\
\hline

Main goal
& The system shall keep track of which customer owns which car and who's the owner of a given car
\\
\hline

Requirement(s)
& The system shall use the customer database to identify the vehicle's serial number based on customer name and vice versa
\\
\hline

Description
& The system keeps track of cars and owners
\\
\hline

Stakeholders
& End-users, developers, customer, testers
\\
\hline

Risk
& The vehicle's serial number is mapped to the wrong customer
\\
\hline

Test type
& System
\\
\hline

Preconditions
& At least one vehicle and customer record with matching keys must be present.
\\
\hline

Input
& \begin{enumerate}
    \item Select a vehicle via serial number
    \item Select a customer based on name search
\end{enumerate}
\\
\hline

Output
& \begin{enumerate}
    \item View updates with vehicle information and associated owner
    \item View selects the owner matching the name and the first matching vehicle based on ownership sorted by lowest index
\end{enumerate}
\\
\hline

Expected result
& Database tracks customers and owners correctly
\\
\hline

Result
& Database tracks customers and owners correctly
\\
\hline

Comments
& Everything works as expected, though extra care needs to be taken when a customer has more than one car, as only the first one is selected.
\\
\hline

Passed
& Yes
\\
\hline

\end{tabularx}
\end{table}

\clearpage