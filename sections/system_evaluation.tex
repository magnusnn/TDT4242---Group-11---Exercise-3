\section{System Evaluation}

This section will give an evaluation of the SoCam system.

\subsection{Evaluation of the systems functionality}
After going through the tests designed for the system, the results were quite good. Only one test did not pass in the Factory-system. However the code coverage did not show that the system as a whole is used. The group was quite surprised that the coverage was not higher than showed in \autoref{sec:code_coverage}. But we did not dig into the code to see why this was the case as this was outside the scope of this exercise. 

There were also some functionalities that were not mentioned in the requirements, such as adding customers, removing customers, removing vehicles, and sending customers emails from the garage system. The latter two do not seem to work, but were not a part of the requirements (judging by the exercise documents) and for that reason not tested by the group. 

\subsection{User experience}

Although most of the requirements are implemented and working, the graphical user interface could use an improvement. The factory panel is an unpleasant user experience due to the lack of information on how to operate the system, and the feedback the user receive is either non-existent or poor. The user interface can be confusing at first. One example of a bad user experience is the initiate recall panel, which could be improved by adding a drop-down menu instead of having to type in the numbers. There are also some typos in the GUI that should be corrected. The garage system suffers many of the same flaws seen in the factory system; namely an unintuitive and obfuscated interface. Adding descriptive tool tips and better grouping of buttons and actions will aid the user in properly navigating the different functions. Some typos were also found.


\subsection{Final conclusion and shipment recommendation}

All but one requirements that were tested managed to pass the test on the current build of the software. Although the system seem to be in a good state, the lacking requirement should be implemented before release as it can be the cause of frustration for both the customer and the end-user. It is also important to improve the user interface before shipment as well as looking into the code coverage of the system to see if there is some lack of functionality that is not used. Lastly the functionalities that are not derived through the requirements should either be fixed or removed. Priority should be placed on the functionalities that are derived from the requirements though. \\

\noindent Prioritized list  of defects to be corrected:

\begin{enumerate}
    \item Requirement 1.4 - Update action scripts 
\end{enumerate}

\noindent To conclude, there are some issues with the software that should be fixed before shipment. 

